% Options for packages loaded elsewhere
\PassOptionsToPackage{unicode}{hyperref}
\PassOptionsToPackage{hyphens}{url}
%
\documentclass[
]{article}
\usepackage{amsmath,amssymb}
\usepackage{lmodern}
\usepackage{ifxetex,ifluatex}
\ifnum 0\ifxetex 1\fi\ifluatex 1\fi=0 % if pdftex
  \usepackage[T1]{fontenc}
  \usepackage[utf8]{inputenc}
  \usepackage{textcomp} % provide euro and other symbols
\else % if luatex or xetex
  \usepackage{unicode-math}
  \defaultfontfeatures{Scale=MatchLowercase}
  \defaultfontfeatures[\rmfamily]{Ligatures=TeX,Scale=1}
\fi
% Use upquote if available, for straight quotes in verbatim environments
\IfFileExists{upquote.sty}{\usepackage{upquote}}{}
\IfFileExists{microtype.sty}{% use microtype if available
  \usepackage[]{microtype}
  \UseMicrotypeSet[protrusion]{basicmath} % disable protrusion for tt fonts
}{}
\makeatletter
\@ifundefined{KOMAClassName}{% if non-KOMA class
  \IfFileExists{parskip.sty}{%
    \usepackage{parskip}
  }{% else
    \setlength{\parindent}{0pt}
    \setlength{\parskip}{6pt plus 2pt minus 1pt}}
}{% if KOMA class
  \KOMAoptions{parskip=half}}
\makeatother
\usepackage{xcolor}
\IfFileExists{xurl.sty}{\usepackage{xurl}}{} % add URL line breaks if available
\IfFileExists{bookmark.sty}{\usepackage{bookmark}}{\usepackage{hyperref}}
\hypersetup{
  pdftitle={TDE 1 - Análise de complexidade de código não-recursivo},
  pdfauthor={Marcello Fabrizio},
  hidelinks,
  pdfcreator={LaTeX via pandoc}}
\urlstyle{same} % disable monospaced font for URLs
\usepackage[margin=1in]{geometry}
\usepackage{graphicx}
\makeatletter
\def\maxwidth{\ifdim\Gin@nat@width>\linewidth\linewidth\else\Gin@nat@width\fi}
\def\maxheight{\ifdim\Gin@nat@height>\textheight\textheight\else\Gin@nat@height\fi}
\makeatother
% Scale images if necessary, so that they will not overflow the page
% margins by default, and it is still possible to overwrite the defaults
% using explicit options in \includegraphics[width, height, ...]{}
\setkeys{Gin}{width=\maxwidth,height=\maxheight,keepaspectratio}
% Set default figure placement to htbp
\makeatletter
\def\fps@figure{htbp}
\makeatother
\setlength{\emergencystretch}{3em} % prevent overfull lines
\providecommand{\tightlist}{%
  \setlength{\itemsep}{0pt}\setlength{\parskip}{0pt}}
\setcounter{secnumdepth}{-\maxdimen} % remove section numbering
\ifluatex
  \usepackage{selnolig}  % disable illegal ligatures
\fi

\title{TDE 1 - Análise de complexidade de código não-recursivo}
\author{Marcello Fabrizio}
\date{7/9/2021}

\begin{document}
\maketitle

Este é um R Notebook que contém as minhas soluções para as complexidades
dos trechos de código dados como questões.

\hypertarget{trecho-1}{%
\subsubsection{Trecho 1}\label{trecho-1}}

\begin{verbatim}
for (i=0;i<n;i++) {
   for (j=i+1;j<n;j++) {
      if (m[i,1]>m[j,1]) {
         for (k=0;k<n;k++) {
           // Ops. Fundamentais
           aux=m[i,k];
           m[i,k]=m[j,k];
           m[j,k]=m[i,k];
         }
      }
   }
}
\end{verbatim}

No trecho acima, considerei as três operações fundamentais como uma
única só para facilitar os cálculos. Na opeção \texttt{for} mais
interna, teremos que ela executuá \textbf{n} vezes. Para o próximo
\texttt{for}, com sua dependência pela variável \textbf{i}, temos que a
quantidade de suas execuções será de

\[\sum_{i=0}^{n-1} n-(i+1)\] Agora, podemos tornar o somatório acima em
\(\sum_{i=0}^{n-1} n\) - \(\sum_{i=0}^{n-1} i\) -
\(\sum_{i=0}^{n-1} 1\), e assim obtemos

\[(n²-n)/2\]

como resultado da complexidade do trecho 1.

\hypertarget{trecho-2}{%
\subsubsection{Trecho 2}\label{trecho-2}}

\begin{verbatim}
for (i=0;i<n;i++) {

      for (k=n-1;k>=i;k--) {
           m[i][k]/=m[i][i]; <- op fund
      }
      
      for (k=i+1;k<n;k++) {
            for (j=n-1;j>=i;j--){
                  m[k][j]=m[k][j]/m[k][i]-m[i][j]; <- op fund.
            }
      }

}
\end{verbatim}

Para o primeiro \texttt{for} interno obtemos a complexidade como
\[\sum_{i=0}^{n-1} n-1\] Simplificando este resultado teremos
\[(n^2 + n)/2\]

E para o segundo e terceiro laço interno, teremos que a complexidade de
ambos será \[\sum_{i=0}^{n-1} (n-(i+1))*(n-i))\] Então teremos a
complexidade do trecho 2 como

\[\sum_{i=0}^{n-1} n-1 + \sum_{i=0}^{n-1} (n-(i+1))*(n-i))\] (Não
consegui encontrar uma fórmula simplificada para a somatória da segunda
complexidade)

\hypertarget{trecho-3}{%
\subsubsection{Trecho 3}\label{trecho-3}}

(considere a operação \(log\) como \(log_2\))

\begin{verbatim}
void Moo(int N)
{
      for (j = 1; j <= N; j = j * 2)
            for (i = j; i > 1; i = i / 2)
                  printf("%d\n", i);
}
\end{verbatim}

É possível re-organizar os laços do treco da seguinte maneira:

\begin{verbatim}
void Moo(int N)
{
      for (i = 1; i <= N; i = i * 2)
            for (j = 2; j <= i; j = j * 2)
                 // op fundamental
}
\end{verbatim}

Para resolver a complexidade do trecho de código acima, como suas
variáveis de controle não são incrementadas for um a cada iterção,
\textbf{i} e \textbf{j} serão substituidas por \(2^y\) e \(2^k\)
respectivamente. Assim teremos

\begin{verbatim}
void Moo(int N)
{
      for (2**y = 1; 2**y <= N; y++)
            for (2**k = 2; 2**k <= 2**y; k++)
                  // op fundamental
}
\end{verbatim}

E podemos simplificar ainda mais, chegando ao trecho final: (considere)

\begin{verbatim}
void Moo(int N)
{
      for (y = 0; y <= logN; y++)
            for (k = 1; k <= log2**y; k++)
                  // op fundamental
}
\end{verbatim}

Dessa forma teremos o laço interno como
\[\sum_{k=1}^{log_2N}log_22^k = \sum_{k=1}^{log_2N}k\] E temos o laço
externo como \[log_2N\] Podemos simplificar o resultado que obtivemos do
laço interno para \[(log_2N + 1)/2\] e então multiplicamos pelo
resultado do laço externo para chegarmos a equação de complexidade
\[((log_2N)^2 + log_2N)/2\]

\end{document}
